
\chapter{Conclusions}

\section{Research Area:}

\begin{itemize}
    \item Assistive technology dedicated to facilitating communication for non-verbal disabilities is still in the research phase - although extensive - and none of the existing data gloves translating sign language has moved into structured testing 
    \item Data gloves prototypes which translate sign language uses expensive hardware components which makes them hard to afford for people who need it 
    \item Existing Sign language gloves are very bulky and expensive and not durable or long lasting with no proper testing to document feedback of performance and always need to be paired with a device to operate 
    \item None of the existing sign language data gloves have plans to take the projects further into production and no studies have been made into the readability of the innovation market to be disrupted in this field 
    \item No working prototype of a sign language glove to translate Arabic sign language or Makaton sign language (used by autistic children) to text or speech has ever been made
    \item There is no universal sign language and different disabilities have their own variation of Sign Language which means any version of a sign language translation glove is limited to cater to one form of sign language 
\end{itemize}

\section{Research Scope:}

\begin{itemize}

    \item Use computation innovation to eliminate communication barriers between people with different disabilities
    \item Give a voice to those who can’t speak
    \item Facilitate communication between people with speech disabilities and the general public
    \item Make innovation accessible to improve the life of people with disabilities 
    \item Make assistive technology for communication affordable so everyone can use as an extension of their senses 
    \item Design an affordable and durable solution to translate sign language, detach from existing expensive hardware and resort to reducing hardware size and components by designing a flexi-circuit board, enhancing software, and pairing with machine learning
    \item Design a wireless and stand-alone data glove which operates independently from any device
    \item Allow users to record their own sign language hand gestures to the glove  
    \item Catering to different disabilities on both ends of communication: Sign Language speakers and listeners/receivers (Mute: use sign language to replace speech, Deaf: people use sign language to speak to them - also use sign language to speak, Blind: can communicate with speech disabled to hear through the speaker)

\end{itemize}

\begin{quote}\textbf{
    \large Research Questions:
    \small
    \begin{itemize}
        \item{RQ1: How can an assistive technology innovation be made to facilitate communication for people with disabilities who use sign language for their daily communication?}
        \item{RQ2: How can this innovation be made universal and compatible with different libraries of sign language?}
        \item{RQ3: How can this innovation be made more accessible and more affordable so that it can be more easily available as a commercial product?}
    \end{itemize}
}\end{quote}

\section{Proposed Research Outcomes} 

\begin{itemize}
    \item To design a robust high quality single board computer device with an arm processor which runs Linux system 
    \item The research will be divided in two main parts: 
    \begin{enumerate}
        \item Prototype Design
        \item Usability studies and analysis
    \end{enumerate}
    \item Research Methodologies: Interaction Research employing user-centred design methods and Grounded Theory in a build-measure-learn iteration.
    \item Explorative studies of physical tangible interaction with domain
    \item Multiple studies on low cost accessibility tools
    \item Introduce the prototypes to explore the extent of the research, focusing on hands sensor based interaction with gestures (sensor based data gloves)
    \item Recruit real participants with speech disability and children with non-verbal autism who use sign language as the main mean of communication to test the prototypes of the sign language glove - integrate their feedback in the loop of prototype design development 
    \item Use smart machine learning to record and recognize hand gestures to allow users to upload their personalized versions of sign language on their gloves 
    \item Design a web application for storing and sharing new hand gestures 
    \item Focus on stand alone - wireless - mobile - affordable - accessible - durable design 
\end{itemize}