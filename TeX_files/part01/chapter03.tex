\chapter{Research Methods}

Human-problem oriented inventions\footcite{Cox2008} , similar to my proposed design of the data glove, have conventionally employed user-centred design research methods\footcite{Bevan1999}. Rather than starting with an idea for a system based on what technology can do, and then trying to determine whether people will be able and willing to use it, instead I will start with people's needs and ability; and find a technology that they will be able to use to fulfil a need. This strategy is confirmed in multiple research resources in HCI\footcite{Dix2004} and is referred to as ‘Interaction Design’.

The main steps in such a strategy are the following: 
\begin{enumerate}
    \item Identify a problem that requires a solution, which then becomes the research goal. This can be confirmed through surveys, interviews and observation. 
    \item Find the source of the problem. What is causing the difficulty?
    \item Invent a solution to help people with their difficulty. This can be done through multiple rounds of testing and developing to prove that the proposed solution is valid. Interaction design research will be used in this phase to develop and test iterations of the designed system, in a build-measure-learn loop.
    \item Create a system which incorporates findings and make it available to the people who struggled with the previously identified problem. If the function people wanted to perform but couldn't do well is made more available, chances are it will be successful, considering of course it is affordable, which is a major consideration in this research.
\end{enumerate}

For the theoretical part of my research, Grounded theory will be employed, where theory emerges from the collected data which will be gained through multiple rounds of data collection. Interaction design method will then be applied to use the collected data to prove the theory through a series of usability studies which will mainly constitute of demonstration case studies followed by longitudinal \& In-depth case studies. This will be discussed in detail in the evaluation chapter.
I propose combining Grounded Theory with Interaction Design Research, to structure the iterative design phase of Grounded Theory. I chose Interaction Research because it is more of a holistic approach to problem-solving, rather than a single method for collecting and analysing data\footcite{OBrienRoryFacultyofInformationStudies2001}. In conjunction with this, user-centred design usability testing will then be used to validate the proposed solution for the identified problem. 

\section{Grounded Theory} 

Grounded Theory started as the analysis of qualitative research data. However, it was later identified as ``a method of qualitative research that aims to produce new theories that are grounded in the qualitative data gathered during the research'' \footcite{Glaser1967}.
Researchers Strauss and Corbin\footcite{Strauss1990} used the term ``Grounded Theory'' to refer to a theory building approach based on an analysis technique they formulated of collected data that can incorporate both qualitative data sets such as interviews, focus groups, observations and ethnographic studies and quantitative data sets such as questionnaires, logs and experimental data. ``The research findings constitute a theoretical formulation of the reality under investigation, rather than consisting of a set of numbers, or a group of loosely related themes''\footcite {Strauss1990}.

Grounded theory is different from other research methods in that it does not require a prior hypothesis before conducting research\footcite{Glaser1967}. Researchers may approach the research with an identified goal without knowing what they expect to find\footcite{Adams1997}. ``The process of doing the research formulates the theory and therefore produces potential hypotheses for further study''\footcite{Adams1997}. A side-effect of this is that research data previously collected on the same phenomena can be used for further research.

In Grounded Theory, the theory is developed once there is available data to analyse and not once the data collection phase is concluded. A good example is the first interview, although one interview is not sufficient to base a theory on, it is however a good indication of validating and expanding the theory and leads to a tentative theory\footcite{Cox2008}.
In subsequent interviews, the researcher would design the questions with the intention of testing the limits of the theory. The second interview analysis would either confirm or reject the theory. It may even produce a new theory and so on. ``Thus, the method proceeds through cycles of data gathering, analysis and theorizing''\footcite{Cox2008}. As a result, interview questions progress from the initial interview and are generated based on the results of each cycle of interviews. Questions can be very different later in the study than the very first interview. This approach is applied to different data collection methods throughout the study where the reliability of the method is tested through ``systematic repetition of observations in quantitative research''\footcite{Strauss1990}.
Strauss and Corbin\footcite{Strauss2008} suggest that grounded theory is especially useful for complex subjects or phenomena where little is yet known. This is a major reason why I have chosen it as the main research method for conducting this research. ``The methodology's flexibility can cope with complex data and its continual cross-referencing allows uncovering previously unknown issues''\footcite{Strauss2008} and grounding of the theory specifically to validate the proposed solution for the identified. Emphasis is placed on theoretical sampling and contextual considerations so that later transferability of finding can be increased. 
This is useful for new emerging fields of research relevant to innovation and assistive technology. 
The collected data is analysed in a standard grounded theory format. It is then broken down, conceptualized and put back together in new ways. To enable this to occur in a structured manner, Strauss and Corbin\footcite{Strauss2008} have devised three major bonding stages – ``open, axial and selective''\footcite{Strauss2008} - in the analysis procedure. The lines between these forms of coding are artificial, as is the division between data collection and analysis.  This is an analytic distinction, but in practice, all of these elements of grounded theory analysis intersect as the interpretation proceeds.

Coding categories in my research as identified in the research question are:
\begin{itemize}
    \item \textbf{Assistive:} Effective in facilitating daily communication between sign language users and the public. This specifically measures the performance of the glove, its durability and comfort and mobility.
    \item \textbf{Universal:} Can be used by adults (all genders), children, output different languages, compatible with any platform, translate different libraries of sign language including customized gestures.
    \item \textbf{Accessible:} Can be made available to people who need it, not requiring any external hardware or device, stand alone, wireless.
    \item \textbf{Affordable:} Cost effective – reasonably priced.
\end{itemize}

The reason I chose to pair grounded theory with interaction design research as my proposed research methodology is because both grounded theory and interaction design research methods employ an ``interplay between data collection and data analysis, which results in the concepts and theory truly emerging from the data''\footcite{Lazar2010}. In this approach, detailed and through coding is conducted from the multiple rounds of data collection. Results depends on researchers listening to the data. As in most research in the HCI field, both text-based information and multimedia-based information will be collected from the participants\footcite{Lazar2010}. 
However, since I will be designing new technology and studying speech-based interaction, I will also need to evaluate a number of issues relating to the recognition rate, which requires comparison between the recorded data and the system output. 
This leads me to the description of kinds of methodology that bear on the invention of new computer-based methods\footcite{Rogers2011} and to be combined with the coding criteria for grounded theory, specifically to evaluate system performance: 

Failure Analysis: is to find out specifically where things go wrong. 
Individual Difference Analysis: This is to identify that certain kinds of users, ones with certain background characteristics or abilities, affect the results of testing the system in different ways. This is directly relevant to my proposed universal design of a sign language data glove to be used by all ages, genders, languages and abilities. 
Time Profiling: Time profiling is used to measure and analysing how much time Is spent on isolated tasks within the system. Time profiling is important in identifying problems in the system and potential areas for improvement\footcite{Cox2008}. 

\section{Interaction Design Research: build - measure - learn}

Referring to multiple resources on research methods in HCI\footcite{Cox2008} \footcite{Dix2004} \footcite{Lazar2010} \footcite{Zimmerman:2007:RTD:1240624.1240704}, I have identified the practice portion of my PhD proposal as interaction design. The start and focus of any interaction design is the intended user or users\footcite{Dix2004}. The user in my case is speech disabled individuals who use sign language for their daily communication. My research, design and evaluation will be based on their needs. Consequently, testing rounds will employ user-centred design research methods.

In principle, interaction design research is ``learning by doing'': researchers identify a problem, design a solution, test and evaluate their proposal, and if not satisfied, try again using the feedback they gained from the research cycle.  While this is the essence of the approach, there are other key attributes of interaction design research that differentiate it from other problem-solving research methods. One being its emphasis is on scientific study. In interaction design research, the problem is studied systematically, and intervention is informed by theoretical considerations, which in my case will be the outcome of grounded theory research.  In Interaction design research, data is presented on an ongoing basis. All the while, the methodological tools are being refined to suit the demands of the research\footcite{OBrienRoryFacultyofInformationStudies2001}. 
 
Another reason that I chose Interaction design research, is because it is a user-cantered research methodology. Interaction design research focuses on turning the people involved in the studies and testing into researchers, too. ``People learn best, and more willingly apply what they have learned, when they do it themselves.  It also has a social dimension - the research takes place in real-world situations, and aims to solve real problems''\footcite{OBrienRoryFacultyofInformationStudies2001}. This is the exact setting for my research studies, where real participants will test and use the data glove, sometimes over a long period of time and mostly in their own environments.
The interaction design process\footcite{Dix2004} of the research will be divided into four main phases plus an iteration loop (feeds evaluations back into the deign), focused on the design of interaction, illustrated in Figure \ref{fig:idpdiagram}.

\begin{figure} 
    \centering
    \begin{tikzpicture}[node distance=1em, >=stealth', bend angle=45, auto, every node/.style={scale=0.7}]
    
        \tikzstyle{place}=[rectangle, rounded corners, thick, draw=blue!22, fill=blue!20, minimum width=5em]
        
        \tikzstyle{every label}=[red]
        
        \begin{scope}
            \node [place] (id)								{Identify a Problem};
            \node [place] (an) [right=of id]               	{Analysis};
            \node [place] (de) [right=of an, yshift=3em]	{Design a Solution};
            \node [place] (pr) [below=of de, yshift=-3em]   	{Prototype};
            \node [place] (ev) [right=of de, yshift=-3em]           		{Test \& Evaluate};
            \node [place] (im) [right=of ev]               	{Implement \& Deploy};
            
            \path (id) edge[->] (an);
            \path (an) edge[->, bend left=20] (de);
            \path (de) edge[->, bend left=20] (ev);
            \path (ev) edge[->, bend left=20] (pr);
            \path (pr) edge[->, bend left=20] (an);
            \path (ev) edge[->] (im);
        \end{scope}

        \begin{pgfonlayer}{background}
            \filldraw [line width=4mm,join=round,black!10]
            (de.north  -| id.west) rectangle (pr.south  -| im.east);
        \end{pgfonlayer}
    \end{tikzpicture}
    \caption{Iterative Design Process; based on Figure 4.1 of Dix et al., 2004}
    \label{fig:idpdiagram}
\end{figure}

\begin{quote}
\textbf{Requirements:} The first stage is establishing what exactly is needed. As a pioneering study in this field it is necessary to find out what is currently happening. For example, how do speech disabled individuals currently interact in public using sign language? How does the process of communication work? 
A number of techniques have been documented to be used for this in HCI\footcite{Dix2004} \footcite{Zimmerman:2007:RTD:1240624.1240704} like interviews, video documentation and direct observation.
 
\textbf{Analysis:} Observation and interview are analysed to highlight how people carry out various tasks in relation to the problem identified. The results are classified in a format to outline key issues resulting in task models. Task analysis methods are then developed and applied to formulate a proposal for a design solution. 

\textbf{Design:} Design is at the core of the interaction design process. This phase starts with the data gathered from previous steps and moves from what we need to design, to how we should design. Design loops are then attempted based on user testing and feedback, in compliance with user-centred design principles. 

\textbf{Iteration and prototyping:} Evaluation of prototypes will be based on usability testing feedback. Observations will be made in terms of performance and improvement areas. Most user interface designs involve some form of prototyping, producing early versions of systems to try out with real users\footcite{Bevan1999}. This is my approach for the proposed data glove design. Prototyping iteration will be discussed in more detail in evaluation methods. 

\textbf{Implementation and deployment:} Finally, when the design gives indications that is successful based on user feedback from testing rounds, the plan is to create it and deploy it. This will involve finalising writing code, concluding hardware design, writing documentation and manuals - everything that goes into a real system that can be given to others in preparation for production. 
\end{quote}
