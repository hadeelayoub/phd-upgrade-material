1.	Case Studies - Literature Relating to Chosen Methods  

This research is user-centred. It is therefore based entirely on case studies. It is important to highlight the goals of HCI case studies (Lazar, Feng and Hochheiser, 2010)(Cox and Cairns, 2008) and the role they play feeding straight into interaction design research:
•	Exploration: Case studies provide valuable feedback in understanding novel problems especially in the early phases of the research.  Results often set the foundation for further investigation to inform new system design. 
•	Explanation: Case studies of tools are used to understand a context of the proposed technology.  It is very common in computer systems that study participants use the technology in unexpected ways that were not considered in the initial design which impacts the iterative design loop (Klasnja, Consolvo and Pratt, 2011).     
“As HCI researchers often use a case study as a tool for understanding the technology usage and needs of populations of potential users, HCI case studies often largely draw upon representative users and use cases, omitting extreme cases.” (Lazar, Feng and Hochheiser, 2010)
•	Description: Descriptive case studies are longitudinal and in-depth case studies.  They contribute to documenting a system, a context of technology use, and the process that led to a proposed design. They are particularly useful for technology involving new design methodologies. In interaction  research, the process behind the design is usually the focus of the case study.  “Case studies that describe design processes and results have been written for a wide variety of topics in HCI, specifically for participants with impairments.” (Lazar, Feng and Hochheiser, 2010) (Cox and Cairns, 2008).   

•	Demonstration: Demonstrative case studies are shorter and less in-depth than descriptive case studies.  Their purpose is to show how a new tool was successfully used.  Participants demonstrate the effective use of a new tool to complete one or more assigned tasks.  

Case studies in this research will be of two types: Demonstration case studies followed by descriptive longitudinal and in-depth case studies.

5.1 Demonstrative Case Study 

A good example is a case study conducted by Shinohara and Tenenberg (Shinohara and Tenenberg, 2009) of a blind person’s (Sara) use of assistive technology. Sara’s case study focused on one person’s use of technology. How a blind person might use a variety of assistive technologies to achieve tasks, user interactions, including failures and response to those failures.
In this case study, Shinohara and Tenenberg (Shinohara and Tenenberg, 2009) used three types of technology biography (Blyth and Mon and Park, cited (Cox and Cairns, 2008)): “demonstrations of devices (technology tours), reflections on memories of early use of and reactions to devices(personal histories), and wishful thinking about possible technological innovations (guided speculation)”(Shinohara and Tenenberg, 2009). Data sources used in this study demonstrate three types of case study data: “artefacts, observation, and interviews” (Shinohara and Tenenberg, 2009).
A total of 12 hours was recorded in Sara’s home, broken down into six, two hour sessions.  Raw data consisted of written notes, audio recordings, interviews and photo documentation.  Twelve tasks were defined and recorded in terms of their goals.  The insights from the individual tasks guided the design of improved tools (Shinohara and Tenenberg, 2009).
Although Sara does not provide a comprehensive picture of the needs and concerns of all blind people, the investigations of her needs and goals led to valuable insights that might apply to many other blind people.  The Shinohara and Tenenberg (Shinohara and Tenenberg, 2009) case study helped the researchers to understand how Sara used a variety of technologies to accomplish multiple tasks.  They were specifically interested in understanding “what technologies were most valued and used, when they were used and for what purpose” (Shinohara and Tenenberg, 2009). Conducting the study in Sara’s home helped the investigators gain insights into how she actually addressed real challenges, as opposed to the more engineered results that might have been seen in the lab.  
Sara’s case study demonstrates four key aspects used to describe case studies for users with impairments.  These points align with my research methods and will be followed as guidelines in the case studies of my research:
•	In-depth investigation of a small number of cases: In-depth, broad examinations of a small number of cases are used to address a vast range of concerns.  

•	Examination in context: Labs have the advantage of removing undesired external influences which is not a realistic or credible environment to show how the technology would work. On the other hand, single case studies conducted in a realistic context give meaningful results which are applicable in the real world and are more informative than large scale case studies conducted in a lab.

•	Multiple data sources: Known as data triangulation and is especially important in single case studies. Multiple data sources are combined to validate the evidence and the quality of the data. Contradictions are important too because they compel the researcher to dig deeper, consulting new data sources, which is the essence of grounded theory and action research.

•	Emphasis on qualitative data and analysis: Question of how the technology was used to achieve an assigned task are more important than how long it took to complete it. Researchers focus on the quality of the system in successfully delivering what is was designed for rather than the system speed. 
It is important to highlight that although single case studies can be very informative about the success of a system, results cannot be generalized to include all members of user criteria especially in disability. The real value of single case studies lie in creating realistic insights into design challenges which can be applied to a broader scale of users. 
“Sara’s case study led to some suggestions for the design of assistive devices that would help Sara with her daily challenges, but could go further, to influence insights that apply to many blind people.  As a result, designs might be useful to a much broader range of blind users.” (Shinohara and Tenenberg, 2009)
The goal of Sara’s case study was: a deeper understanding of a blind user’s use of assistive technology in her home.  Similarly, usability case studies in this research will have a centre goal of understanding speech disabled participants’ use of the data glove and how effective it is in facilitating their daily communication and interaction within a public setup. 

5.2 Descriptive Longitudinal and In-Depth Case Study

In depth case studies executed in-context, in realistic environments, present credible and valuable evidence.  Careful consideration is given to the selection criteria of case study participants. Analyzing the data from the case studies and further interpretation is of the upmost importance [Yin, (Lazar, Feng and Hochheiser, 2010)].
In these studies, the process of developing a new system or interaction technique is more important than the end product, especially for innovations that tackle new challenges in the context of use (Cohene et al., 2007)
A study at the University of Toronto (Cohene et al., 2007) provided the base for a very interesting single in-depth case study involving the design of an assistive technology tool to help people with Alzheimer’s disease. “This project was based in a body of prior work that firmly established the importance of reminiscences for people with Alzheimer’s disease.” (Cohene et al., 2007) The goal of the case study was to develop a multimedia tool to help people with Alzheimer’s disease recall and relive old memories. The sole participant of the case study was a 91 year old woman named Laura. Laura and her two daughters was fundamental in the study which focused on developing a system to help Laura with her memory. (Cohene et al., 2007) 
The study started with an exploratory phase to understand Alzheimer’s disease challenges faced by patients and their families. A broad understating of the disease was necessary even though the study was aimed to develop a tool specifically tailored to the needs and abilities of Laura.  Researchers’ observations resulted in a comprehensive understanding of the “abilities and impairments of the participants, leading to a set of design principles” (Cohene et al., 2007). The study also included feedback from caretakers and therapists which acted as a basis in outlining a set of guidelines to assist with memory recollection. As part of the study, family members were required to complete a “family workbook” accumulating stories in the form of pictures, videos and music.  The collected media was to be included in the tool the researchers were working on developing, with the main purpose of helping the study participants with Alzheimer’s disease remember.   
The tool was developed through a series of prototypes which lead to an interactive multimedia device informed by the system whit output displayed on a screen.  The prototypes were refined based on the feedback of the study participants during eight testing sessions over a period of four weeks (Cohene et al., 2007). 
The research team conducted follow-up interviews with family members which confirmed that the system contributed in enhancing the memory of the participants. 
“This project as a whole is an exploratory case study. As relatively little work has been done on user interfaces for people with Alzheimer’s disease, the description of a successful process is valuable in and of itself” (Cohene et al., 2007). The proposed design served to generate further investigations rather than as a solution. 
It is very hard to generalize when it comes to disability and especially a cognitive one like Alzheimer’s disease. Researchers on this case study aimed at extending the applicability of this work by scaling the design process to include more participants to improve the tool(Cohene et al., 2007).
This research required serious time commitment from all parties involved: participants with Alzheimer’s disease, their family members, and research team members. This, combined with the emotional strain, required intensive resources.  Even though the result could not be generalized to other users, the documentation of the design process and the resulting designed tool were considered important contributions (Cohene et al., 2007).

“The most broadly applicable results from this story lie in the lessons learned. The authors concluded that new design methods and principles were needed for working with individuals with Alzheimer's disease, that active participation was more stimulating than passive, and that working with both the patients and their family members throughout the entire design process was necessary. Practical concerns included the resource-intensive nature of the research, the emotional commitment required of the family members, the need to make the approach practical for larger numbers of families, and the need for standards for evaluation” (Cohene et al., 2007).
Although drawn from this particular project, these insights might be extremely valuable to others interested in conducting related research.
Similar to this study, my research will require working directly with speech disabled participants and children with non-verbal autism.  My research dictates interacting with family members, therapists and caregivers of case study participants.  Also, through the process of testing and collecting information, I can learn a lot about the nature of the disability and how my design of an assistive tool can help not only the participants but also the broad spectrum of users with similar disabilities making my design proposal universal and accessible to many people.

